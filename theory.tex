\documentclass[11pt]{article}

\usepackage{amsmath,amssymb,amsfonts}
\usepackage{geometry}
\usepackage{hyperref}
\usepackage{booktabs}
\usepackage{mathtools}
\usepackage{microtype}
\usepackage{enumitem}

% Glossary helpers with superscript numbering in order of first appearance
\makeatletter
\newcounter{glsctr}

% In-text superscript marker K#
\newcommand{\gls@mark}[1]{\textsuperscript{\smash{\scriptsize K#1}}}

\newcommand{\gls@init}[1]{%
  \expandafter\ifx\csname glsnum@#1\endcsname\relax
    \expandafter\gdef\csname glsnum@#1\endcsname{0}%
  \fi
  \expandafter\ifx\csname glsname@#1\endcsname\relax
    \expandafter\gdef\csname glsname@#1\endcsname{#1}%
  \fi
}

\newcommand{\gls@ensure}[1]{%
  \gls@init{#1}%
  \edef\gls@tmp{\csname glsnum@#1\endcsname}%
  \ifnum\gls@tmp=0\relax
    \stepcounter{glsctr}%
    \edef\gls@tmp{\arabic{glsctr}}%
    % Store expanded digits globally (not a macro token)
    \expandafter\xdef\csname glsnum@#1\endcsname{\gls@tmp}%
  \fi
}

\newcommand{\glsid}[1]{%
  \gls@ensure{#1}%
  \gls@mark{\csname glsnum@#1\endcsname}%
}

\newcommand{\glsref}[1]{\glsid{#1}}

\newcommand{\glsx}[2]{%
  \gls@ensure{#1}%
  #2\,\gls@mark{\csname glsnum@#1\endcsname}%
}

\newcommand{\gls}[1]{%
  \glsx{#1}{\csname glsname@#1\endcsname}%
}

% Register only (assign number and store name; no typeset output)
\newcommand{\registerglossaryitem}[2]{%
  \gls@ensure{#1}%
  \expandafter\gdef\csname glsname@#1\endcsname{#2}%
}

\newcommand{\glossaryitem}[2]{%
  % Ensure number assignment on first appearance
  \gls@ensure{#1}%
  % Store display name
  \expandafter\gdef\csname glsname@#1\endcsname{#2}%
  % Fetch number
  \edef\thisglsnum{\csname glsnum@#1\endcsname}%
  % Heading with inline prefix (no superscript) e.g. K3 Term
  \paragraph{\textbf{K\thisglsnum\ \ #2}}\label{gls:#1}%
}
\makeatother

% Pre-assign glossary K-numbers in appendix order so numbering starts at K1
% and remains consistent across the document.
\begingroup
\registerglossaryitem{CET1}{CET1 (Common Equity Tier 1)}
\registerglossaryitem{RWA}{RWA (Risk-Weighted Assets)}
\registerglossaryitem{CET1RWA}{CET1/RWA ratio and slack}
\registerglossaryitem{Capacity}{Credit capacity / Headroom ($V_C$)}
\registerglossaryitem{Pressure}{Credit pressure ($p_C$)}
\registerglossaryitem{HQLA}{HQLA (High Quality Liquid Assets)}
\registerglossaryitem{LiquidityBuffer}{Liquidity buffer}
\registerglossaryitem{LCR}{LCR (Liquidity Coverage Ratio)}
\registerglossaryitem{NSFR}{NSFR (Net Stable Funding Ratio)}
\registerglossaryitem{MoneyIn}{Money-in-circulation}
\registerglossaryitem{CreditStocks}{Credit stocks and flows}
\registerglossaryitem{MarginCredit}{Margin credit / Securities financing}
\registerglossaryitem{CP}{Commercial Paper (CP)}
\registerglossaryitem{PolicyWork}{Policy work ($W_{\mathrm{policy}}$)}
\registerglossaryitem{MECE}{MECE (Mutually Exclusive, Collectively Exhaustive)}
\registerglossaryitem{StateVariable}{State variable and proxy}
\registerglossaryitem{Hysteresis}{Hysteresis and loop area}
\registerglossaryitem{StressTesting}{Stress testing and early-warning indicators}
\registerglossaryitem{MoneyVsCredit}{Money vs.\ credit (QTM vs.\ QTC perspective)}
\endgroup

\geometry{
  a4paper,
  margin=25mm
}

\title{Thermo-Credit Theory: A Credit-First Thermodynamic Mapping}

\author{ToppyMicroServices O\"U}
\date{Version 1.0 -- November 6, 2025}
\hypersetup{
  pdfauthor={ToppyMicroServices O\"U},
  pdfborder={0 0 0},
  colorlinks=true,
  linkcolor=black,
  citecolor=black,
  urlcolor=blue
}


\begin{document}
\maketitle

\begin{abstract}
We recast the Quantity Theory of Credit (QTC) in a thermodynamic-style state space
to separate scale, dispersion, and capacity in modern credit systems.
The key step is an entropy-like dispersion index
that factors money-in-circulation and its allocation,
combined with an explicit capacity variable for bank balance sheets.
This mapping is an \emph{analytic isomorphism}, not a physical identity:
it is designed for structured bookkeeping, stress diagnostics,
and early-warning indicators, not for importing physical laws into economics.
Within this framework we define an internal potential $U(S_M,V_C,\dots)$,
derive a first-law-like decomposition of credit creation,
introduce a Helmholtz-style free energy $F_C$ and an exergy-like measure $X_C$,
and obtain a Maxwell-like integrability condition that makes the mapping empirically falsifiable.
The construction is deliberately practice-first:
all quantities are intended to be computable from public data
and to support decision-useful monitoring for investors,
risk managers, and policymakers.
\end{abstract}

\section{Introduction}

Classical money-first views such as the Quantity Theory of Money (QTM)
explain nominal dynamics in terms of a money stock, its velocity,
and shocks or policy actions \cite{Fisher1911,Friedman1956,Laidler1985}.
More recent credit-first views, notably the Quantity Theory of Credit (QTC),
emphasize that banks create deposits when they lend
and that the use of credit (real vs.\ financial, productive vs.\ speculative)
matters for macro-financial outcomes \cite{Werner2012,BoE2014}.
In parallel, both information theory and statistical mechanics
have inspired analogies for income and wealth distributions \cite{Theil1967,Foley1994,DragulescuYakovenko2000,Shannon1948,Jaynes1957}.

This note takes a minimal and operational step:
we build a thermodynamic-style state description
 that (i) separates scale and dispersion of monetary/credit uses,
(ii) introduces an explicit capacity variable for bank balance sheets,
and (iii) yields testable constraints and early-warning gauges.
The aim is not to claim that macro data obey physical laws,
but to provide a disciplined bookkeeping analogy
useful for supervision, \glsx{StressTesting}{stress testing}, and systematic monitoring.

\section{Setup and Definitions}

We work over a chosen system (jurisdiction, sector set)
and time aggregation (e.g.\ monthly).
Let $M_{\mathrm{in}}$ denote \glsx{MoneyIn}{money-in-circulation} over this domain,
and let $\{q_i\}$ be a stable, \gls{MECE} partition of its uses
(real activity, housing credit, securities margin, etc.),
with $\sum_i q_i = 1$.

\subsection{Monetary dispersion entropy}

We define an entropy-like extensive index
\begin{equation}
  S_M \;=\; k\,M_{\mathrm{in}}\,H(q),
  \qquad
  H(q) \equiv -\sum_i q_i \log q_i,
  \label{eq:SM}
\end{equation}
where $k>0$ is a conventional scaling constant.
This mirrors ideal mixing entropy
$\Delta S_{\mathrm{mix}} \propto N H(x)$
and follows the information-theoretic form of Shannon \cite{Shannon1948}.
$S_M$ grows with both scale and dispersion;
concentration (reallocation into fewer uses) reduces $H(q)$
and can reduce $S_M$ even at fixed $M_{\mathrm{in}}$.

\subsection{Capacity and conjugate variables}

We introduce:
\begin{itemize}[leftmargin=1.5em]
  \item $V_C$: an effective \glsx{Capacity}{credit capacity or headroom}
    (e.g.\ \glsx{CET1RWA}{CET1/RWA} slack, \gls{HQLA}-based lending space, \glsx{LiquidityBuffer}{liquidity buffers}).
  \item $T_L$: a liquidity-intensity index
        (a ``temperature-like'' proxy from spreads, turnover, depth).
  \item $p_C$: a ``\glsx{Pressure}{credit pressure}'' (shadow price of capacity).
\end{itemize}

We postulate an analytic potential
\begin{equation}
  U = U(S_M, V_C, \ldots),
\end{equation}
and define conjugate quantities
\begin{equation}
  T_L \;\equiv\;
  \left( \frac{\partial U}{\partial S_M} \right)_{V_C},
  \qquad
  p_C \;\equiv\;
  -\left( \frac{\partial U}{\partial V_C} \right)_{S_M}.
\end{equation}
Here $U$ is not a physical internal energy;
it is a bookkeeping potential chosen so that a first-law-like decomposition holds.
Classical QTM/QTC do not name $U$; we introduce it as an analytic device.

\section{Thermodynamic Correspondence (QTM vs.\ QTC)}

Table~\ref{tab:correspondence} summarizes the analogy.
Thermodynamic quantities are literal on the left column;
QTM/QTC entries are analogy-level constructs.
Heat- and work-like pieces are defined for bookkeeping only.

\begin{table}[t]
\centering
\small
\begin{tabular}{llll}
\toprule
Variable & Thermodynamics & QTM (money-first) & QTC (credit-first) \\
\midrule
Mixing entropy &
$\Delta S_{\mathrm{mix}}$ &
$S_M = k M_{\mathrm{in}} H(q)$ &
$S_M = k M_{\mathrm{in}} H(q)$ \\
Temperature &
$T$ &
$T_L$ (liquidity proxy) &
$T_L$ (liquidity proxy) \\
Internal energy &
$U$ &
$U_M(S_M)$ (auxiliary) &
$U(S_M,V_C,\dots)$ \\
Volume &
$V$ &
-- &
$V_C$ (capacity/headroom) \\
Pressure &
$p$ &
-- &
$p_C = -(\partial U/\partial V_C)_{S_M}$ \\
Heat-like term &
$\delta Q_{\mathrm{rev}} = T\,dS$ &
$Q_M \sim T_0\,\Delta S_M$ &
$Q_C \sim \bar T_L\,\Delta S_M$ \\
Work-like term &
$\delta W = p\,dV$ &
$W_M \equiv W_{\mathrm{policy}}$ &
$W_C \equiv -\bar p_C \Delta V_C + W_{\mathrm{policy}}$ \\
First law &
$\Delta U = T\Delta S + W$ &
$\Delta U_M = T_0\Delta S_M + W + \varepsilon$ &
$\Delta U = \bar T_L \Delta S_M + W + \varepsilon$ \\
Second-law tendency &
$\Delta S \ge 0$ &
$\Delta S_M \ge 0$ (mixing) &
$\Delta S_M \ge 0$ (mixing) \\
Helmholtz free energy &
$F = U - TS$ &
$F_M \equiv U_M - T_0 S_M$ &
$F_C \equiv U - T_0 S_M$ \\
Exergy/availability &
$X$ &
$\approx U_M - T_0 S_M$ &
$X_C = \Delta U + p_0\Delta V_C - T_0\Delta S_M$ \\
\bottomrule
\end{tabular}
\caption{Thermodynamic--QTM--QTC correspondence (analogy-level).}
\label{tab:correspondence}
\end{table}

The construction is chosen so that:
(i) $S_M$ is extensive and sensitive to dispersion;
(ii) $V_C$ carries explicit capacity/headroom information;
(iii) QTC---unlike classic QTM---naturally admits a pressure-like channel $p_C$.

\section{Bank Credit Creation and the Energy-Like Balance}

When a bank grants a new loan, it simultaneously creates a matching deposit.
On the joint bank--customer system, this is a pure accounting operation
(loans create deposits).
Within our mapping, the associated change in the potential $U$ is decomposed as
\begin{equation}
  \Delta U \;\approx\;
  \bar T_L\,\Delta S_M
  \;-\; \bar p_C\,\Delta V_C
  \;+\; W_{\mathrm{policy}}
  \;+\; \varepsilon,
  \label{eq:firstlaw}
\end{equation}
where:
\begin{itemize}[leftmargin=1.5em]
  \item $\Delta S_M$: change in dispersion entropy from
        scale and allocation shifts (typically $\Delta S_M > 0$ in net expansion);
  \item $\Delta V_C$: change in effective capacity
        ($\Delta V_C < 0$ when headroom is used up; then $-\bar p_C \Delta V_C > 0$);
  \item $W_{\mathrm{policy}}$: structured \glsx{PolicyWork}{policy work} from regulation, guarantees,
        central bank operations, and other deliberate interventions;
  \item $\varepsilon$: residual term capturing model error and omitted channels.
\end{itemize}

Credit creation phases (new lending $>$ repayments) and contraction phases
(repayments $>$ new lending) then exhibit characteristic sign patterns
for $(\Delta M_{\mathrm{in}}, \Delta S_M, \Delta V_C, -\bar p_C\Delta V_C)$,
which can be tabulated and compared with data.
The value of~\eqref{eq:firstlaw} is diagnostic:
it forces us to attribute changes either to dispersion, capacity use,
or \glsx{PolicyWork}{policy work}, rather than conflating them.

\section{Free Energy, Exergy, and Early-Warning Gauges}

We seek a scalar gauge that (i) decreases as dispersion rises
under a fixed environment, and (ii) upper-bounds structured work
extractable over a cycle. A Helmholtz-style free energy provides this.

For a fixed reference environment $(T_0)$,
define
\begin{equation}
  F_C \equiv U - T_0 S_M.
  \label{eq:FC}
\end{equation}
Then, under suitable regularity conditions,
\begin{equation}
  dF_C = -\,p_C\,dV_C + \delta W_{\mathrm{other}},
\end{equation}
so that $F_C$ plays the role of an available-potential measure.
As $\Delta F_C \to 0$ under chosen boundaries,
policy headroom for structured adjustments is effectively exhausted.

When an ambient pressure-like parameter $p_0$ is also relevant,
an exergy-like functional is
\begin{equation}
  X_C
  = (U-U_0)
    + p_0 (V_C - V_{C0})
    - T_0 (S_M - S_{M0}),
  \label{eq:XC}
\end{equation}
which reduces to $-\Delta F_C$ if $V_C$ is fixed and $p_0$ effects are negligible.
Because $X_C$ depends on boundary choices $(T_0,p_0)$,
we treat it as an optional, environment-dependent gauge.

A Gibbs-style free energy
\begin{equation}
  G_C \equiv U + p_0 V_C - T_0 S_M
\end{equation}
can be used when $(T_0,p_0)$ are the natural controls.
In applications, we typically monitor $F_C$ (fixed environment)
and compare with $X_C$ or $G_C$ when capacity constraints are explicit (cf.\ exergy notions in \cite{Wall1977}).

\section{Maxwell-Like Relation and Falsifiability}

Since $U$ is assumed to be a bona fide state potential,
mixed partial derivatives commute.
Using the definitions of $T_L$ and $p_C$,
we obtain a Maxwell-like condition:
\begin{equation}
  \left(\frac{\partial T_L}{\partial V_C}\right)_{S_M}
  = -\left(\frac{\partial p_C}{\partial S_M}\right)_{V_C}.
  \label{eq:maxwell}
\end{equation}
Empirically, one can estimate $T_L(S_M,V_C)$ and $p_C(S_M,V_C)$
from proxies and test whether~\eqref{eq:maxwell} approximately holds.
Systematic violations indicate that the chosen variables
are not state-like or that proxies and specifications are inadequate.
In this sense, the analogy is falsifiable: it is not mere rhetoric.

\section{Insights and Practical Tests}

The mapping yields several operational diagnostics:
\begin{enumerate}[label=(\roman*),leftmargin=1.5em]
  \item \textbf{Integrability test.}
        Estimate $T_L$ and $p_C$ as functions of $(S_M,V_C)$
        and test~\eqref{eq:maxwell}.
        Failure suggests mis-specification of capacity or dispersion metrics.
  \item \textbf{Work vs.\ dispersion decomposition.}
        Use~\eqref{eq:firstlaw} to decompose changes in $U$
        into heat-like dispersion ($\bar T_L \Delta S_M$),
        capacity use ($-\bar p_C \Delta V_C$),
        and policy work $(W_{\mathrm{policy}})$.
  \item \textbf{Free-energy and exergy ceilings.}
        Monitor $F_C$ or $X_C$ as early-warning gauges:
        sustained proximity to zero under stress scenarios
        flags limited room for non-disruptive adjustment.
  \item \textbf{Loop area and \glsx{Hysteresis}{hysteresis}.}
        Compute loop areas in $(S_M,V_C)$ over policy cycles.
        Non-zero areas capture dissipative stress and irreversibility
        in credit allocation dynamics.
\end{enumerate}

All of these rely on observable or reconstructible quantities
from public balance-sheet and market data.
Implementation details (exact proxy choices, normalization, robustness checks)
are part of ongoing empirical work and should be reported alongside results.

\section{Limitations and Scope}

This construction is intentionally modest.
Key limitations include:
\begin{itemize}[leftmargin=1.5em]
  \item \textbf{Category dependence.}
        $S_M$ depends on the chosen partition of uses.
        Robustness across reasonable partitions must be checked.
  \item \textbf{Proxy noise.}
        $T_L$, $V_C$, $p_C$ are built from noisy proxies.
        Measurement error can break~\eqref{eq:maxwell}
        and distort the decomposition~\eqref{eq:firstlaw}.
  \item \textbf{Quasi-static approximation.}
        Fast crises and regime shifts violate smooth state-variable assumptions;
        the mapping is best seen as quasi-static or coarse-grained.
  \item \textbf{Non-physical.}
        We do not assume microscopic ``money particles''
        or claim that macro-financial data obey physical first/second laws.
        The mapping is analogy-level and judged only by empirical usefulness.
  \item \textbf{Identification challenges.}
        Policy, expectations, and shocks act jointly.
        Causal claims require careful research design
        beyond the bookkeeping framework presented here.
  \item \textbf{Scaling conventions.}
        Constants such as $k$ are conventional.
        Report normalized metrics and sensitivity checks.
\end{itemize}

\section{Notes on Reproducibility and AI Assistance}

This note is designed to accompany code and data
(e.g.\ monthly reports computed from public sources).
A typical implementation:
\begin{itemize}[leftmargin=1.5em]
  \item computes $S_M$ from $M_{\mathrm{in}}$ and a stable \gls{MECE} partition of flows;
  \item constructs $V_C$ from capital ratios, liquidity metrics, and headroom;
  \item builds $T_L$ from market microstructure indicators;
  \item estimates $p_C$ via regressions or event-study style differentials;
  \item evaluates the decompositions and tests described above.
\end{itemize}

Drafting and editing of the present text used a GPT-based large language model
as a \emph{tool} at low randomness.
All equations, definitions, and claims are curated and are the responsibility
of the authors.
Readers should rely on the archived PDF and associated repository
for the citable version and reproducible code.

\section*{Disclaimer}

This material is part of an ongoing research program.
It is provided ``as is'' without any warranty of accuracy,
completeness, or fitness for a particular purpose.
It does not constitute investment, legal, tax, or policy advice,
and does not create a client relationship or any regulated financial service.
Users are responsible for compliance with applicable laws
in their jurisdiction.

\appendix
\section*{Appendix: Glossary (For Physicists New to Finance)}
    extit{Numbering: Each term is assigned a unique K-number (K1, K2, \dots) in the order it is first referenced in the main text. Use the superscript K\# markers to jump back here; headings show the same inline K\#.}

This glossary explains the main banking, credit, and regulatory terms
used in this note for readers with a physics background.
Definitions are informal and intuition-friendly.
When helpful, we include analogies to thermodynamics or dynamical systems.
These are \emph{aids to intuition}, not strict identities.

\subsection*{1. Regulation and Bank Balance Sheets}

\glossaryitem{CET1}{CET1 (Common Equity Tier 1)}\footnote{Glossary terms carry a superscript K-number (K1, K2, \dots) at their first appearance. K\# indexes the ordered list in the Appendix Glossary for fast lookup.}
Core equity capital of a bank (common shares, retained earnings).
It absorbs losses and underpins solvency.  
\textit{Analogy: the bank’s fundamental ``energy reserve''.}

\glossaryitem{RWA}{RWA (Risk-Weighted Assets)}
Total assets weighted by regulatory risk factors.
Riskier exposures receive higher weights.
The ratio CET1/RWA is a key prudential metric.  
\textit{Analogy: an effective load or mass adjusted for fragility.}

\glossaryitem{CET1RWA}{CET1/RWA ratio and slack}
CET1/RWA indicates how much high-quality capital backs risk-weighted assets.
Regulation sets a minimum.  
The \emph{slack} is the excess above this minimum and represents capacity to expand or absorb shocks.  
\textit{Analogy: safety margin or unused headroom in a constrained system.}

\glossaryitem{Capacity}{Credit capacity / Headroom ($V_C$)}
Effective room for additional lending, given capital, liquidity, and internal limits.
In this note $V_C$ is the state-like variable capturing this remaining capacity.  
\textit{Analogy: effective volume available before hitting hard walls.}

\glossaryitem{Pressure}{Credit pressure ($p_C$)}
Shadow price of capacity:
the marginal cost or value of relaxing or tightening $V_C$.
Defined in analogy to
$p_C = -(\partial U / \partial V_C)_{S_M}$ or $-\partial F / \partial V_C$.  
\textit{Analogy: thermodynamic pressure; tighter constraints $\Rightarrow$ higher $p_C$.}

\glossaryitem{HQLA}{HQLA (High Quality Liquid Assets)}
Very liquid, low-risk assets (e.g.\ government bonds) that can be sold quickly in stress.
They support liquidity ratios and crisis resilience.  
\textit{Analogy: high-grade stored energy that can be tapped on demand.}

\glossaryitem{LiquidityBuffer}{Liquidity buffer}
Cash plus HQLA held as an emergency reserve
to withstand sudden outflows or market stress.  
\textit{Analogy: a backup battery or safety reservoir.}

\glossaryitem{LCR}{LCR (Liquidity Coverage Ratio)}
Regulatory ratio:
HQLA divided by projected net cash outflows over a 30-day stress scenario.
It tests short-term liquidity robustness.  
\textit{Analogy: can the system run for 30 days under worst-case load?}

\glossaryitem{NSFR}{NSFR (Net Stable Funding Ratio)}
Regulatory ratio over a one-year horizon:
stable funding relative to required stable funding.
It limits excessive maturity mismatch.  
\textit{Analogy: ensuring long-lived assets are backed by sufficiently slow-decaying sources.}

\subsection*{2. Money, Credit, and Flow Concepts}

\glossaryitem{MoneyIn}{Money-in-circulation}
An operational measure of money that is actually circulating
in the selected system and horizon (e.g.\ deposits used in payments),
not merely base money on the central bank balance sheet.  
\textit{Analogy: active particles taking part in interactions.}

\glossaryitem{CreditStocks}{Credit stocks and flows}
\emph{Stocks}: outstanding amounts of loans or credit at a given time.  
\emph{Flows}: changes per period (new lending, repayments, net issuance).  
\textit{Analogy: internal energy vs.\ power / energy flux.}

\glossaryitem{MarginCredit}{Margin credit / Securities financing}
Credit used to finance positions in securities (e.g.\ margin loans, certain repos).
It amplifies leverage in financial markets and is often separated
from credit to the real economy.

\glossaryitem{CP}{Commercial Paper (CP)}
Short-term unsecured debt issued by firms.
Provides funding outside traditional bank loans.  
\textit{Analogy: an alternative branch in the credit circuit.}

\glossaryitem{PolicyWork}{Policy work ($W_{\mathrm{policy}}$)}
Deliberate interventions by central banks, regulators, or governments:
interest rate changes, asset purchases, guarantees,
capital rule adjustments, and similar actions.
In our decomposition this is the structured, intentional part of ``work''.  
\textit{Analogy: externally applied work on the system.}

\subsection*{3. Structural and Analytical Concepts}

\glossaryitem{MECE}{MECE (Mutually Exclusive, Collectively Exhaustive)}
A rule for defining categories:
(i) no overlaps between categories,  
(ii) no gaps in coverage.  
For the entropy $S_M = k M_{\mathrm{in}} H(q)$,
the allocation shares $q_i$ should be MECE,
otherwise $S_M$ is distorted by double counting or omissions.

\glossaryitem{StateVariable}{State variable and proxy}
A \emph{state variable} is determined by the current state,
not by the detailed path.
In practice, finance uses observable \emph{proxies}
(e.g.\ ratios, indices) to approximate such variables.
Our Maxwell-like relation tests whether chosen proxies
behave as if they came from a consistent potential $U(S_M,V_C,\dots)$.

\glossaryitem{Hysteresis}{Hysteresis and loop area}
If trajectories in the $(S_M, V_C)$ plane form loops with non-zero area
over policy or credit cycles,
this indicates irreversibility or dissipation-like effects
(e.g.\ stress, misallocation, or path dependence).  
\textit{Analogy: magnetic hysteresis, frictional cycles, or inelastic processes.}

\glossaryitem{StressTesting}{Stress testing and early-warning indicators}
Stress tests simulate severe but plausible scenarios
to see whether banks or systems survive.
Early-warning indicators attempt to flag fragility in advance.
Within this framework, quantities like $F_C$, $X_C$,
and loop areas are intended as structured candidates
to complement such tools.

\glossaryitem{MoneyVsCredit}{Money vs.\ credit (QTM vs.\ QTC perspective)}
In traditional Quantity Theory of Money (QTM),
money is primary and credit is often treated as derived.
In the Quantity Theory of Credit (QTC) and in this note,
\emph{credit creation is primary}:
bank lending creates deposits,
and money is an accounting outcome of credit decisions.
This shift motivates tracking credit capacity and dispersion
as state-like objects.

% Include uncited works that were previously listed manually
% Suppress bibliography heading caption
\renewcommand{\refname}{}
\bibliographystyle{unsrt}
\bibliography{references}

\end{document}